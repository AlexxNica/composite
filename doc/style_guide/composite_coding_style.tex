\documentclass[11pt,onecolumn]{article}
%\usepackage{latex8}
%\usepackage{epsfig,graphics}
\usepackage{times}
%\usepackage{floatflt}
\usepackage{amsmath}
\usepackage{paralist}
\usepackage{boxedminipage}
\usepackage{epsfig,graphics}
%%%%\pagestyle{empty}

\usepackage{enumitem}
%\setlist{nolistsep}
%%\setlist{leftmargin=1.5em}
%%\setlist{parsep=0em}
%%\setlist{topsep=0em}
%\setlist{itemsep=1px}
\usepackage{wrapfig}
\usepackage{listings}
\usepackage{hyperref}
\hypersetup{
  colorlinks=true,
  citecolor=blue,
  linkcolor=blue,
  urlcolor=blue,
  citebordercolor={1 1 1},
  linkbordercolor={1 1 1},
}
%\usepackage[hmargin=1in,vmargin=1in]{geometry}
\usepackage[hmargin=1in,top=1in,bottom=1in]{geometry}

%\usepackage{fullpage}		%tmp

%Needed to correct for margin error -- RW
%\setlength{\topmargin}{0.625in}

%\setlength{\oddsidemargin}{0.1in}
%\setlength{\evensidemargin}{0.0in}
%\setlength{\textheight}{9.8in}
%\setlength{\textwidth}{6.5in}
%\renewcommand{\baselinestretch}{.92}

\def\gindent{\phantom{X}}
\def\composite{\textsc{Composite}}
\def\superglue{\textsc{Superglue}}
\def\sidl{\textsc{sIDL}}
\def\glue{\textsc{Glue}}
\def\cbuf{\textsc{CBuf}}
\def\cbufs{\textsc{CBufs}}
\def\vault{\textsc{Vault}}
\def\relecos{\textsc{RelecOS}}
\def\sysiphus{\textsc{Sisyphus}}
\def\c3{\textsc{C$^3$}}
% http://www.netplaces.com/classical-mythology/the-dark-prince/escaping-death.htm
%crash cart
%phoenix
%defibrillator
%Matilda (Dante)
\def\hires{\textsc{HiRes}}
\def\eg{\textit{e.g.}}
\def\ie{\textit{i.e.}}
\newcommand{\head}[1]{\vspace{0.4em}\noindent{\bf #1}}
\newcommand{\minihead}[1]{\vspace{0.1em}\noindent{\bf #1}}
\newcommand{\lsthead}[1]{{\bf #1}}
\addtocounter{section}{0}
%\addtocounter{page}{-1}


\date{}
\title{\Large\bf The \composite\ Style Guide: \\
a Guide to C and Tips for Systems Programming\\
\author{Gabriel Parmer\\[0.05in]
Computer Science Department \\
The George Washington University\\
Washington, DC
}}

\begin{document}
\thispagestyle{empty}
\maketitle
\thispagestyle{empty}

%\thispagestyle{empty}
%\setcounter{page}{0}

%\input{summary}
%\newpage

%\input{planning}


\begin{abstract}
  This document should be seen as a rough guide first to explain the
  coding conventions and style in
  \composite\footnote{\url{http://www.seas.gwu.edu/~gparmer/composite}}that
  borrows significantly from the Linux style -- for better or worst --
  and second as a set of reasonable conventions and code formatting
  methodologies that can lead to (subjectively) easier to understand
  low-level code.  As with all coding style and convention guidelines,
  they are very subjective; take them or leave them, but take them at
  least as a data-point with some reasoning behind their rules.
  Certainly, if you are adding to, or modifying \composite\ code, you
  should follow these conventions.

  Many of these conventions and tips derive from a very useful
  property of C -- one that drives its use in low-level systems
  programming: If you look at a block of code, there should be no
  ``magic''.  The code itself is self-descriptive, and there is
  nothing happening ``behind the scenes''.  For example, there are no
  hidden memory allocations (\eg\ for closures), no garbage
  collection, and no implicit reference counting or synchronization
  (\eg\ {\tt synchronized} in Java).  The more that C is taken away
  from this transparency, the more you probably shouldn't be using
  C\footnote{One reason that C++ is difficult, is because a given
    block of code can have nearly any meaning, depending on how
    classes are defined, and what syntax is overridden.  This can be
    powerful and if you want this power, use C++, not C.  They say
    that using C is like being given the gun that you can use to shoot
    yourself in the foot, and that C++ is the shotgun.}.

  This document can be seen as one researcher/developer's opinion on
  how to let C be C.  Alternatively, it can be viewed as a set of
  hints for making predominantly self-documenting code.  That is, code
  that is descriptive, and readable on its own.  This is not as an
  excuse to not use comments -- they are required to explain the
  non-obvious.  Instead the focus is on making what should be obvious,
  self-evident from the code itself.  It has been informed by writing
  thousands of lines of code of low-level systems code in
  \composite\ and elsewhere, and from reading many more lines of code
  in a variety of code bases including Linux.

  Please see the Appendix (Section~\ref{s:appendix}) for a terse
  summary of the coding conventions in \composite.

  Currently, this document has organizational problems, so take your
  editor hat off for the time being.
\end{abstract}

The audience for this document is intended to either graduate students
working with me on \composite\ or other projects, or students who are
taking my classes.  If you don't fall into these groups, realize that
the text may be written in a tone and with a certainty that was not
intended for you.

\section{On Reading and Understanding Other's Code}
\label{s:reading}

To understand conventions for {\em writing} code, it is important to
understand how one can productively {\em read} other's code.
Conventions are typically motivated by making code readable and
maintainable.  They are more rarely motivated by making it more
efficient to lay code down onto the screen (programmers usually don't
need discipline to encourage them to take the shortest path from A to
B).  Therefore, this section will discussion how to read code, so that
you can better understand why the following conventions exist.

\begin{enumerate}[itemsep=0pt,topsep=1pt,parsep=1pt,leftmargin=2.5em]
\item \lsthead{Interfaces as the purpose of the code.}  Most systems
  have well defined entry points and exit points at which other parts
  of the system interact.  These interfaces are often quite nuanced,
  and it can take some effort to understand their main purpose.
  Understanding the system interface enables you to understand the
  purpose of the system.  It reveals the main {\em abstractions}
  provided by the system, and the operations involving those
  abstractions.  Without an understanding of the purpose of the
  system, it is difficult to read through the code in a principled
  manner.
\item \lsthead{Data-structures as the maps of the code.}  Systems
  programs are often less related to algorithmic design than they are
  about data-structure design.  Web servers are more concerned with
  tracking sockets, processes, and client requests, and the
  relationships between them, than they are about algorithmic
  complexity.  Data-structures often include quite a bit of
  information about the concurrency and locking structures of the
  program.

  When reading systems code, {\em it is of the utmost importance to
    seek to understand the data-structures and how they are related to
    each}.  Use your understanding of the purpose of the system to
  investigate what the data-structures are meant to record.  Draw
  mental connections between the data-structures and the abstractions
  provided by the interface.  The code is often there to simply glue
  together the data-structures.  It is important to be able to
  understand them to the degree where you can draw them out on paper
  as a block diagram.
\item \lsthead{Data-structure manipulations and operations.}  If the
  data-structures are one of the most important part of the code to
  understand, the second step one should take is to browse through the
  functions that operate on those data-structures and maintain their
  invariants.  This will include the functions that encode the
  interactions between data-structures (these are perhaps the most
  important).

  When browsing through code, you should always ask ``What do I think
  this code will do, and how it will do it'', and only {\em then}
  should you look through the code to either confirm your intuition
  (\ie\ you understand the code), or refute it.  In the latter case,
  you learn more about the system from the differences between what
  you expected from the code, and what is actually there.  This is
  your opportunity to refine your own mental model of how the system
  works.  Here it is important to emphasize that {\em reading code is
    not a passive activity.}  You must always be thinking about what
  you expect from code before you read it, and then confirm or refute
  and refine.  If you simply read code without actively building
  expectations about how things are done, you will simply not
  understand the system for a very long time.
\item \lsthead{Algorithmic complexity and implementation.} Once you
  understand the data-structures, and have a reasonable mental model
  about how they fit together and the invariants on them, then the
  algorithmic glue to bind it all becomes important.  Before you get
  to this level, you should have a good idea about how the code comes
  together to provide an implementation for the interface.  After you
  go through this code, you should have a good idea about how the
  system as a whole fits together, and the trade-offs made therein.
\end{enumerate}

Throughout this process, your concentration should be on refining your
mental model about how the system provides an implementation of the
interface.  Predict how a function will accomplish its goals before
looking at the code based on your understanding of the
data-structures, and refine your understanding where you are wrong, or
oversimplified.

When reading the rest of this document, keep in mind that many of the
conventions are tailored specifically to making this process of others
reading your code easier.  It assumes that they will dig through your
code using this methodology.

\subsection{On Refining Your Mental Model}

It is perhaps easier said than done that you should have a mental
model about what a section of code should be doing, {\em before}
actual reading the code.  How does one go about forming this mental
model?  First, realize that this is the essence of {\em reading} code!
To understand code, means that you understand not trivial things like
why one conditional is formed the way it is, but why the overall code
is structured the way it is.  It is about understanding the underlying
approach and design to approaching the problem at hand.

That said, the devil is in the details, and when you want to {\em
  modify} the code you've been reading, {\em then} it is important to
understand the nitty-gritty in the code.  However, you won't be able
to achieve this understanding if you don't understand the overall
model for how the code fits together.

Building a mental model about a body of code takes practice.  A lot of
practice.  However, I can guarantee that so long as you code in your
future, you {\em will} have to read code.  Likely, you will have to
read tons of code.  This is a skill that is important to acquire.
Additionally, it will make you a better programmer.  Seeing how others
code is very helpful in our personal growth.

To be able to predict what a function does before reading it does rely
on some conventions in the code.  This is one reason why I am
emphasizing conventions in this document.  Well named functions, with
well-named arguments often make their intention self-evident (see
Section~\ref{s:naming}).

\subsection{How ``Deep'' Do You Go?}

One of the most difficult questions you will face when reading
complicated code is which function calls to ``follow''?  This is
actually a very difficult question to answer, and really requires that
you gain an intuition about a code base.  Does the author commonly put
significant functionality into functions?  If so, you have to be more
aggressive in following function calls.

The goal in deciding how deep in function calls to go is ``do I have
to go into this function to understand the common-case path through
this code''?  Section~\ref{s:exception} is devoted to how to make the
common-case code path obvious to readers!

\subsection{How Learn to Read Code -- the Guaranteed Method}

The only method that will guarantee that you can effectively read code
is to {\em read a lot of code}.  It becomes much easier to create a
mental model about the gist of a body of code when you've seen quite a
bit of other code.  In short, experience is the main technique for
increasing code comprehension.  The rest of these conventions will
help significantly in making your code readable, but experience is
immensely helpful.

\section{The Most Important Stylistic Rule}
\label{s:naming}

This is the most important convention.  I will be so assertive as to
call it a truth of programming, a guiding light without which you {\em
  will} produce bad code (as in -- code that is impossible to read and
understand).

\head{Requirement 1.} Choose your names well.  Choose your names to
make an ideal trade-off between descriptive power and terseness.
Names include
\begin{enumerate}[itemsep=0pt,topsep=1pt,parsep=1pt,leftmargin=2.5em]
\item type names (\ie\ from {\tt typedef}s, {\tt struct}s, and {\tt
  union}s),
\item function names,
\item function argument names,
\item variable names.
\end{enumerate}
These have been ordered from the most important to least.  This is not
to say that any of them are unimportant.

\head{Convention 1.} Function names should use a noun-first naming
scheme when possible.  This borrows from object oriented programming.
%(see Section~\ref{s:oo} for more OO in C information).

\subsection{Naming Requirement 1 Justification}
\label{ss:variables}

The summary here is simply, do not blow off or skip on making good
names for types, functions, arguments, or variables.  They are {\em
  the most important} part of your code when it comes to others
understanding and reading it.  ``Self-documenting code'' is code that
does not need comments for someone of sufficient experience to
understand it.  Though there is no such thing as completely
self-documenting code (in any non-trivial code-base, there is some
area that does something complicated that requires a comment), but
that doesn't mean that it isn't a good goal.  This does {\em not} mean
that you just don't write comments.  Many people interpret this as
such.  Instead, it means that you should craft your code such that
writing comments doesn't add much to help people understand it.

\head{Toward writing self-documenting code.}  Proper and intelligent
naming is the main way you can attempt to achieve the goal of
self-documenting code.  Variable names themselves {\em are} comments
in that they convey human readable information about the code.
Data-structure names are exceedingly important.  This include
structure, union, and typedef names.  If data-structures are the maps
of your code, their names are the legend of the map.  Without it,
no-one will have the context to understand the map.  Function names
are also abundantly important as they give a reader the first hint
about what the function does (\ie\ it allows them to predict given
their mental model of the code how the function should be
implemented).  The arguments to the function provide an initial
context about what is important in the function.  {\em A function
  declaration including the function name, argument types and names,
  and return value tell a startling vivid story.}  Do not waste that
opportunity to tell the story.  Variables in conjunction with the
argument names partially document what the body of a function is
doing.  When deciding if you should create a variable, or simply have
longer computation somewhere, the most important question to ask is
``would the new variable's name give more of an indication about what
this function is doing''?  An additional mechanism you have for
writing self documenting functions (in addition to the rest of the
conventions in this document) is to move some of a function's
implementation into separate functions.  This enables you to choose
more function names that can describe the sub-computations of the
function.  For example:

    \begin{minipage}{3in}
      \footnotesize
      \lstset{language=C}
      \begin{lstlisting}
int request_handle(char *request_txt)
{
    struct request req;
    struct response resp;

    request_parse(&req, request_txt);
    request_process(&req);
    request_formulate_resp(&req, &resp);
    request_send_resp(&req, &resp);
}
      \end{lstlisting}
    \end{minipage}

If all of the functions within {\tt request\_handle} were instead the
expanded code, it would be much more difficult to understand what the
function was doing.  However, now it becomes obvious.  There are no
comments, but the code would also not benefit from them in the current
form -- ergo, self-documenting code.

\head{An aside: the {\em onion} model of writing and reading code.}
Good code (\ie\ code written to be read and modified) often exhibits
what we will call the {\em onion} property: When tracking through the
code, any given function should give the reader some understanding of
the computation at a specific level -- or layer -- of detail.  We
start at the functions that define the interface into the system.
They will look much like {\tt request\_handle} above.  There is very
little detail there, but the reader does understand the flow of the
function, and generally what it does.  As with an onion, we can ``peel
off'' a layer of abstraction, and go into more detail by looking at
the functions that are called.  Also like an onion, as you go into
deeper layers, the impulse to cry increases.  The programmer should
realize that a reader does not want to go into {\em all} the gory
details, and protect them from that by providing functions written as
in {\tt request\_handle}.  A reader does not have to dive to the next
layer to understand the function, unless they require that level of
detail.

In the onion model, it is important to not only provide good function
names, but also to know when to push functionality into a separate
function.

\head{Tension between descriptive names, and concision.} There is
always a tension between choosing very descriptive names, and maintain
some concision in your code.  You'll have to use your own experience
here to make the proper trade-off.  Certainly know conventions: {\tt
  i}, {\tt j}, {\tt k} are iterator values.  You should {\em not} use
names other than these as they are concise {\em and} give any
programmer that also understands that common convention as much
information as almost any other variable name you could choose.  It is
often OK to use short form names for variables in functions, so long
as their type is descriptive.  A reader can always look at the type to
better understand the variable.  See {\tt req} and {\tt resp} in the
previous example.  However, remember that humans can store 7 $\pm$ 2
items in their short term memory.  If you rely on this trick for more
than 6 variables, then some readers will have to go back to the
variable declarations very often.  If you do it for 10 variables,
nearly all readers will have to go back to the variable declarations
constantly.  Make their life easier.  However, if you have greater
than 10 variables, you might consider breaking the function up into
sub-functions anyway.

\subsection{Naming Convention 1 Justification}

This convention is suggesting using function names of the form {\tt
  noun\_operation}.  {\tt operation} might be multiple words, but the
important part is that {\tt noun} leads the name.  An example of this
can be seen in the code above.  The noun (\ie\ {\tt request}) is the
object that the function is operating on, and the first argument is
the actual data-structure for that noun ({\tt req}).  This mirrors
object oriented programming %% (see Section~\ref{s:oo})
in which methods
operate on a specific object.  Where-ever you have a set of functions
that operate on a given data structure (see ``Data-structure
manipulations and operations'' in Section~\ref{s:reading}), this is a
good strategy.

Why?  This gives your functions that operate on the data-structure
some visual unity that a reader can immediately identify.  A reader
will realize that the functions have a conceptual unity around the
data-structure, and will enable them to better form their mental
model.

%\subsection{Advice on Naming and Commenting}
%\label{ss:commenting}

\section{Handling Exceptional Cases in Functions}
\label{s:exception}

As you will hopefully see in this section, the following two
conventions are motivated by
\begin{inparaenum}[(i)]
\item keeping the common-case code compact and easy to read, and
\item avoiding redundancy in code, especially error handling code.
\end{inparaenum}

\head{Convention 1.} Make sure that the ``common-case'' code path
through your functions reads from top to bottom, and at the same
indentation level as much as possible.  To achieve this, make your
conditionals be true on {\em error}.

\head{Convention 2.} Use {\tt goto} statements to avoid code
redundancy and handle error cases.  These can and should be used as
exceptions that are local to the function (without all the cruft of
real exceptions).

\head{Convention 3.} Do error handling as soon as you can in a
function.  This is most important for input (function argument) sanity
checking.

\subsection{Exception Cases Convention 1 Justifications}

Find the bug in the following example of code that take a struct and a
list, and make two copies of the struct, adding both into the list.
Assume the list is protected by a lock.

    \begin{minipage}{3in}
      \footnotesize
      \lstset{language=C}
      \begin{lstlisting}
int list_copy_and_dup(struct bar *b, struct list *l)
{
    if (b) {
        if (l) {
            struct bar *b2, *b3;

            list_lock_take(l)
            b2 = malloc(sizeof(struct bar));
            if (b2) {
                memcpy(b2, b, sizeof(struct bar));
                list_add(l, b2);

                b3 = malloc(sizeof(struct bar));
                if (b3) {
                    memcpy(b3, b, sizeof(struct bar));
                    list_add(l, b3);

                    list_lock_release(l);

                    return 0;
                } else {
                    return -ENOMEM;
                }
            } else {
                return -ENOMEM;
            }
        } else {
            return -EINVAL;
        }
    } else {
        return -EINVAL;
    }
}
      \end{lstlisting}
    \end{minipage}

This code could certainly be improved in various ways.  Lets avoid
that now, and use this as an example of two ways to structure a given
piece of code.

Did you find the bug?  If you ran the program on a low-memory system,
you quickly would.  But you have to ask yourself how often you test in
those circumstances.  We forgot to
\begin{inparaenum}[(i)]
\item free {\tt b2} if {\tt malloc} returned {\tt NULL} when attempting
to allocate {\tt b3}, and
\item release the lock if either {\tt malloc} was unsuccessful.
\end{inparaenum}
It is {\em very} easy to make these mistakes.  This is the reason that
languages such as {\tt go}, {\tt D}, and {\tt C++} provide facilities
such as {\tt defer}, {\tt scope}, and {\tt RAII} with automatic
destruction, respectively.  You can have something similar in C, but
you'll have to wait for the second convention.  First, lets reformat
the code under the first convention.

Following this convention, all error handling is dealt with using
conditionals that trigger on error.

    \begin{minipage}{3in}
      \footnotesize
      \lstset{language=C}
      \begin{lstlisting}
int list_copy_and_dup(struct bar *b, struct list *l)
{
    struct bar *b2, *b3;

    if (!b) return -EINVAL;
    if (!l) return -EINVAL;

    list_lock_take(l)

    b2 = malloc(sizeof(struct bar));
    if (!b2) {
        list_lock_release(l);
        return -ENOMEM;
    }

    b3 = malloc(sizeof(struct bar));
    if (!b3) {
        free(b2);
        list_lock_release(l);
        return -ENOMEM;
    }

    memcpy(b2, b, sizeof(struct bar));
    list_add(l, b2);

    memcpy(b3, b, sizeof(struct bar));
    list_add(l, b3);

    list_lock_release(l);

    return 0;
}
      \end{lstlisting}
    \end{minipage}

Shorter and easier to understand.  The main benefit (and you come
appreciate this when you begin to {\em read} lots of code), is that it
is obvious what the common-case path through the code is.  Sometimes
the erroneous case requires significant processing (\eg\ deallocating
a partially allocated list when {\tt malloc} returns {\tt NULL}).  In
these cases, consider moving this computation into a separate function
to avoid it complicating the code flow for the common-case.

This convention does {\em not} say ``avoid levels of indentation for
the common-case path''.  Often you cannot avoid conditionals, and
almost certainly you cannot avoid loops.  This conventions makes a
statement about how to format {\em error cases} that can only clutter
the code.

\subsection{Exception Cases Convention 2 Justifications}

This convention is most useful when used in conjunction with the
previous.  Looking at the previous code, even when following our
convention on how to format the error conditionals, and notice that
there is significant redundancy in the error handling code.  For
example, the code for {\tt list\_lock\_release} is repeated twice.  If
we were to change the error semantics of this code, we might need to
change error code in multiple places.  This is a recipe for disaster,
but can be fixed with a simple convention: use {\tt goto}s as a way to
clean up erroneous cases (similar, as mentioned earlier, to mechanisms
in other languages such as {\tt go}, {\tt D}, and {\tt C++}).

    \begin{minipage}{3in}
      \footnotesize
      \lstset{language=C}
      \begin{lstlisting}
int list_copy_and_dup(struct bar *b, struct list *l)
{
    int ret = -EINVAL; // default error case.
    struct bar *b2, *b3;

    if (!l) return ret;

    list_lock_take(l)
    if (!b) goto err;

    b2 = malloc(sizeof(struct bar));
    if (!b2) {
        ret = -ENOMEM;
        goto err;
    }

    b3 = malloc(sizeof(struct bar));
    if (!b3) {
        ret = -ENOMEM;
        goto err_free;
    }

    memcpy(b2, b, sizeof(struct bar));
    list_add(l, b2);
    memcpy(b3, b, sizeof(struct bar));
    list_add(l, b3);
    ret = 0;
done:
    list_lock_release(l);
    return ret;
err_free:
    free(b2);
err:
    goto done;
}
      \end{lstlisting}
    \end{minipage}

A few things have happened here:
\begin{enumerate}[itemsep=0pt,topsep=1pt,parsep=1pt,leftmargin=2.5em]
\item The return value is now stored in the variable {\tt ret}.  By
  default it is set to the most commonly returned error code (here
  {\tt EINVAL} is the ``invalid argument'' return code -- see {\tt man
    errno}).
\item The normal return conditions (non-erroneous) are annotated by
  the {\tt done} label.  We know that whenever we want to exit the
  function, this is what we should jump to.  Everything between {\tt
    done} and the return must be executed for the common path to exit
  the function.  In this case, the lock must always be released.
\item The erroneous cases are laid out in a ``first error, last
  label'' listed order.  This is intentional, and lets the latter
  errors ({\tt b3}'s {\tt malloc} failing) fall through, and execute
  the return code for the previous error cases.  This essentially
  means that when an error happens, it will always execute the ``clean
  up'' code for the previous errors as well {\em without having to
    duplicate that code anywhere}.  Notice how if there is an error
  allocating {\tt b3}, we free {\tt b2}, and also execute the error
  handlers for everything else (in this case, this simply jumps to
  {\tt done} which releases the lock).
\item The return value for each of the error cases is computed at the
  code location where we know what the error was caused by.  This
  enables the error description to still be as detailed as we want
  (here just returning {\tt -EINVAL} or {\tt -ENOMEM}), while still
  avoiding code replication.
\end{enumerate}

{\em Code replication in error code is the worst type of code
  replication as it can often go untested.}  This convention is
tailored to prevent this from happening by removing the replication.
{\tt goto}s can actually add to the structure of your code!

\head{On program size.}  This convention often creates a few more
lines of code, rather than getting rid of them.  This is a case where
having slightly more code is beneficial as it makes the error handling
more explicit and adaptable to future changes.  Most of the increase
in program size is simply due to labels and gotos, which certainly
isn't adding too much conceptual burden.  Program size is not the
goal; easy to read/understand code, and code that can adapt to future
changes are the goals.

\head{On program efficiency.} All I can say for this is to look at
your compiler's output when using {\tt -O3}.  I have never noticed
this style being appreciably slower than alternative code formats.
Please trust your compiler.  It is smarter than you here.  Make your
code easy to read instead.

\head{Is this encouraging programmers to use {\tt goto}s?} I am
encouraging programs to use {\tt goto} as a structured technique for
exceptional case handling.  {\tt goto}s, like many language devices
can be used very badly, and produce the very definition of spaghetti
code.  Use {\tt goto}s in a structured way to remove error-case code
redundancy.  Do not use them as your new fancy control flow
manipulation mechanism.


\subsection{Exception Cases Convention 3 Justifications}
\label{ss:exceptions}

A reader wishes to distinguish between erroneous cases that are often
not important to understand the ``main gist'' of the code (\ie\ the
mental model), so readers should be given as many hints as possible
that distinguish error checking from the main flow of the code.

A reasonable convention to aid this is to do your error checking as
soon as possible, and where possible cluster error checks together.  A
natural consequence of this convention is that all sanity checking on
function arguments should be done at the top of the function, even if
the arguments aren't used till later\footnote{Sometimes this is not
  possible, as some computation must be done that will enable the
  sanity checking.  This is a convention, not a rule that must always
  be followed.}.  This rule, used in conjunction with those earlier in
this section to make it very obvious what the common path through the
code is.

\section{{\tt typedef} Usage and Naming}

    \head{Convention 1.} Do {\em not} use {\tt typedef} when the type
    synonym being created is 1) a composite type ({\tt struct} or {\tt
      union}), or 2) an array.  Use {\tt typedef} to create types only
    when it adds context to a variable declaration that will help the
    programmer.  Do not use {\tt typedef}s just to avoid typing {\tt
      struct} in your code.  Seeing the {\tt struct} lets the
    programmer know that passing the variable by value will be
    potentially costly.  Seeing a {\tt typedef} should indicate to the
    programmer that they can pass the value directly to functions (and
    across the system call boundary without complications).

    \head{Convention 2.} {\em Never} make a {\tt typedef} to for a
    pointer (\ie\ {\tt typedef int * ptrtypdf}) as its correct use
    will violate normal visual norms for accessing and setting the
    variable.

    \head{Convention 3.} Always create {\tt typedef} names with a
    trailing {\tt \_t} (\eg\ {\tt pthread\_t}).

\subsection{{\tt typedef} Convention 1 Justification}

    When looking at a block of C code, one should be able to
    distinguish without much further investigation certain properties
    about its variables:

    \begin{enumerate}[itemsep=0pt,topsep=1pt,parsep=1pt,leftmargin=1.5em]
    \item If I pass the variable to a function by value, will it pass
      a large chunk of memory, possibly blowing the stack?
    \item If I return this variable from this function, will it return
      in a register, or have to force stack allocation in the calling
      convention?
    \end{enumerate}

    If it is unclear why passing a {\tt struct} around by value
    requires stack allocation, see the {\tt cdecl} C calling
    conventions~\footnote{\url{http://en.wikipedia.org/wiki/X86_calling_conventions\#cdecl}}.

    Long story short, when you look at a variable, you should know if
    you can pass it to functions by value, and if this will cause
    implicit stack allocation or not.  You should know if you can
    return the value from a function, and if you do if it will be
    doing implicit stack allocations.

    \head{{\tt typedef}s gone wrong.}  As with many language features,
    there are good ways to use it, and bad.  In this case, {\tt
      typedef} s make seeing what is happening with stack allocation
    and argument passing difficult in general, as a type name can be
    equivalent to any type declaration.  Take the following code:

    \begin{minipage}{3in}
      \footnotesize
%      \lstset{frame=single,language=C}
      \lstset{language=C}
      \begin{lstlisting}
imatype value;
imafunction(value);
return value;
      \end{lstlisting}
    \end{minipage}

    \begin{enumerate}[itemsep=0pt,topsep=1pt,parsep=1pt,leftmargin=1.5em]
    \item If {\tt typedef int imatype;} in the previous code, then
      nothing out of the ordinary happens.  No memory is allocated
      transparently.
    \item If {\tt typedef struct \{ int array[1024]; \} imatype;}, then we have
      an entirely different case.  Passing a structure by value into
      a function will copy the entire struct, which is expensive and
      takes 4KB worth of memory here.  Even declaring the variable
      value, does the same.  Returning {\tt value} here also causes
      memory allocation and copying (in the calling function).
    \end{enumerate}

    An additional reason to not use {\tt typedef} with composite
    types.  The typedef hides the access mechanism to the variable: Do
    I use {\tt =} to set its value?  Do I use a field in a structure
    as in {\tt value.array}?  As one programming in C usually assumes
    that they can "see" everything that's going on, hiding information
    is not in keeping with consistent C programming.

\subsection{{\tt typedef} Convention 2 Justification}

Using a {\tt typedef} to make a synonym to a pointer type makes code
difficult to understand.  For example,

    \begin{minipage}{3in}
      \footnotesize
      \lstset{language=C}
      \begin{lstlisting}
typedef int * imatype;
...
imatype value;
value->field = x;
      \end{lstlisting}
    \end{minipage}

is simply awkward.  Note that the implicit pointer definition is
hidden by the {\tt typedef}.  This makes common-case code look
abnormal -- dereferencing {\tt value} when it wasn't declared a
pointer.

\subsection{{\tt typedef} Convention 3 Justification}

\head{{\tt typedef} Naming Conventions} All type names created with
     {\tt typedef} should end with a {\tt \_t}.  This allows a reader
     of the code to understand immediately that the name is a type.
     Imagine the use of the following in code.

    \begin{minipage}{3in}
      \footnotesize
      \lstset{language=C}
      \begin{lstlisting}
typedef int u32_t;
typedef int thdid_t;
      \end{lstlisting}
    \end{minipage}

  It becomes obvious that {\tt u32\_t} is a type, avoiding any
  confusion.

As with all variable/type names, the name of a type should be as short
as it can be (types are used in many places in code that can expand a
line to greater than 128 characters), but not too short that its
purpose is unclear.  Clarity is goal \#1.

\subsection{When to use {\tt typedef}s?}

When {\em should} you use {\tt typedef}s?  {\tt typedef}s have two
main purposes:

\begin{enumerate}[itemsep=0pt,topsep=1pt,parsep=1pt,leftmargin=2.5em]
\item For code clarity.  Often having a descriptive type name can make
  code much easier to read.  This is doubly true when you are using
  function pointers extensively.  Creating an appropriate function
  type saves complicated annotations.  For example, compare:

    \begin{minipage}{3in}
      \footnotesize
      \lstset{language=C}
      \begin{lstlisting}
int pthread_create(pthread_t *thread, const pthread_attr_t *attr,
                   void *(*start_routine)(void*), void *arg);
      \end{lstlisting}
    \end{minipage}

    with

    \begin{minipage}{3in}
      \footnotesize
      \lstset{language=C}
      \begin{lstlisting}
typedef void *(*pthd_create_fn_t)(void*);
int pthread_create(pthread_t *thread, const pthread_attr_t *attr,
                   pthd_create_fn_t start_fn, void *arg);
      \end{lstlisting}
    \end{minipage}

    Do note the use of {\tt pthread\_t} and {\tt pthread\_attr\_t} that
    makes the code much more legible.  Also, however, note that it is
    unclear if {\tt thread} or {\tt attr} are structs, or primitive
    types (to use Java terminology).

    When making a {\tt typedef} for a function, I find a useful
    convention to be to end the type name with {\tt \_fn\_t}, so that
    it is more obvious what the type designates.
  \item For portability.  Often the concrete type behind a typedef
    should be different depending on the platform being used.  For
    instance, a {\tt u64\_t} should be a {\tt unsigned long} on a 64
    bit machine, or an {\tt unsigned long long} on a 32 bit machine.

    It can be used as a form of restricted polymorphism, but I'll
    leave that for another time.
\end{enumerate}

%% \subsection{When to Ignore the Conventions.}

%% As with any convention, it is not always the correct option in all
%% circumstances.  For ``normal'' code, I find the conventions are
%% appropriate.  A case of exceptional code:

%%     \begin{minipage}{3in}
%%       \footnotesize
%%       \lstset{language=C}
%%       \begin{lstlisting}
%% typedef union {
%%         cbuf_t v;
%%         struct {
%%                 u32_t id:26, idx:6;
%%         } __attribute__((packed)) c;
%% } cbuf_unpacked_t;
%%       \end{lstlisting}
%%     \end{minipage}

%% Here I want to be able to pass a data item around as a primitive type
%% (\ie\ guaranteed to be in registers, or on the stack according to {\tt
%%   cdecl}) without stack copying (in fact, it is passed through a
%% kernel code path that allows only registers to persist -- thus the
%% entire value must fit in a register).  {\em However}, I also want to
%% access the value like a bit-field.  Different bits in the value have
%% specific meanings.

\section{Macro Usage}
\label{s:macros}

Macros are one area in C that can completely destroy the {\em
  transparency} of C.  When looking at a block of code, it can
literally do almost anything given specific macros.  However, macros
really are the best solution (available in C) for some problems.
Instead of specific recommendations here, I'll make high-level
suggestions, and you'll have to hone your own judgment as to if the
proposed macro is tasteful or not.

\head{Suggestion 1.} Only use macros when they will reduce code
duplication significantly.  We use extensive macros in \composite\ to
define the system call interaction layer.  This layer must include
inline assembly that is mostly shared across different system calls,
and the generated code only differs in the number of arguments passed
to a call.  We use macros to generate the stubs for each system call,
thus avoiding replication of the inline assembly code ({\tt libc} uses
macros in a similar way).  In Linux, the linked list implementation
uses extensive macros to help provide the equivalent of an iterator
for simply looping through the list (\ie\ {\tt for\_each}).  This
avoid duplication of code at each loop site.  Macros are C's only
alternative to C++'s templates.  They can reduce code in the same way
by inlining type definitions into {\tt struct}s and code (\eg\ see
{\tt SGLIB}).

\head{Suggestion 2.} Assume that someone reading your code can lookup
the macro definition.  Make the macro easy to read, or comment it if
it is doing something truly crazy.  Make it clear in the code when a
macro is being used, so that programs can treat this as a ``red flag''
-- something tricky going on here.  The traditional convention is to
use all caps for any macro definition, though this does not always
hold true (see {\tt for\_each} described above).

\head{Suggestion 3.} Do {\em not} use macros where simple inline
functions will suffice.  Functions carry type information that can be
useful for someone reading and using the code, whereas macros often
only acquire type context when placed into the code.

\head{Macros vs. {\tt enum}s} Macros are still often used for constant
values.  This tradition is so entrenched in C culture, that it isn't
worth fighting against.  You {\em will} see code that uses macros for
constant values, so you should get used to it.  However, some instead
use {\tt enum}s for constant values (see plan 9 source code).

Compare:

    \begin{minipage}{3in}
      \footnotesize
      \lstset{language=C}
      \begin{lstlisting}
#define THD_RUNNING 1
#define THD_KILLED  2
#define THD_BLOCKED 3
      \end{lstlisting}
    \end{minipage}

with

    \begin{minipage}{3in}
      \footnotesize
      \lstset{language=C}
      \begin{lstlisting}
enum {
  THD_RUNNING = 1,
  THD_KILLED  = 2,
  THD_BLOCKED = 3
};
      \end{lstlisting}
    \end{minipage}

Even better, compare these with

    \begin{minipage}{3in}
      \footnotesize
      \lstset{language=C}
      \begin{lstlisting}
typedef enum {
  THD_RUNNING = 1,
  THD_KILLED  = 2,
  THD_BLOCKED = 3
} thd_state_t;
      \end{lstlisting}
    \end{minipage}

When you are defining constants that are part of a larger conceptual
whole (\ie\ different values that describe the state of a thread),
then you should use an {\tt enum}, and as the thread state will
probably be stored in a Thread Control Block (TCB), you might as well
name the ``larger conceptual whole'' with a {\tt typedef}.  This is an
instance of using typedefs to give program readers more context to
understand your code.

Macros are sufficient, however, if they name a constant that is not
part of a larger conceptual whole.  For example

    \begin{minipage}{3in}
      \footnotesize
      \lstset{language=C}
      \begin{lstlisting}
#define DEBUG_ON 1
      \end{lstlisting}
    \end{minipage}

or

    \begin{minipage}{3in}
      \footnotesize
      \lstset{language=C}
      \begin{lstlisting}
#define LOWEST_KERNEL_ADDRESS 0xc0000000
      \end{lstlisting}
    \end{minipage}

These are one of the ways that they are commonly used.  Though it can
certainly be argued that even in these cases one should use {\tt const
  int DEBUG\_ON = 1;} instead, I have not had a sufficient number
experiences where this additional descriptive information was more
significant than simply sticking with the C traditions.

\section{Formatting}
\label{s:formatting}

A ``religious war'' that will never end because it is driven mainly by
subjective opinions is one of formatting and indentation.  Instead of
coming down strongly on either side of any argument here, I want to
posit some general principles and goals, and analyze how different
means of formatting your code can aid those goals.

\head{Goal 1.} We want the formatting itself to encourage good
practices, and discourage bad.

\head{Goal 2.} We want the formatting to aid the reading process.
This is made difficult by the subjective nature of ``correct
formatting''.  Your reader might not agree with your style, but the
formatting can still aid their reading.

\head{Variable declaration location.} c99 enables you to do variable
declarations inline with code, as opposed to at the top of blocks (as
in C++ and most other languages).  Though the original purpose of
variables being declared at the top of blocks, was likely due to
old, constrained C compilers.

There are arguments that can go either way:
\begin{enumerate}[itemsep=0pt,topsep=1pt,parsep=1pt,leftmargin=2.5em]
\item Variables should be declared at the top of the scope block.  All
  variables are in one place, so it is easy to reference them (see
  discussion on variables names in Section~\ref{s:naming}), and there
  is a predictable progression about the code.  When using this style,
  It is often best to declare variables in the inner-most block
  possible (\ie\ inside for loops, and conditional blocks) especially
  when the inner block is an erroneous condition.  This enables the
  reader to know by looking at an enclosing block what the common-case
  variables are, and not those that exist to handle the error
  cases.\label{lst:top}
\item Declare variables where they are used.  This is very appealing
  as the first use has the luxury of being very close to the
  declaration.  It has the additional benefit that you can use {\tt
    const} in more situations.  However, for the second, third, and
  20th use of the variable, it can be difficult to backtrack through
  the function and find out where it is defined.  It is this lack of
  predictability and regularity in the code that makes this option not
  a clear win.
\end{enumerate}

I prefer option \ref{lst:top}, as I find it provides more structure to
the code, and makes it more predictable to read.  This is the style
\composite\ uses.  However, I find it breaks down when you have large
functions, and it can be difficult to find the top of a block.
However, we probably shouldn't have that long of functions anyway,
should we (Section~\ref{ss:variables})?

\head{Use of blank lines.} Where should you use blank lines in your
code?  First, I'll comment simply on many versus few blank lines.
Though it might be nice to see your code spaced out (\ie\ with a blank
line between each line of code), it is much more preferable for it to
be compact.  If a reader has a strong preference for spaced out code,
it isn't that difficult to increase the space between lines in their
editor.  However, when code is compact (without blank lines), more of
it can fit on the screen, which makes it easier to read as you can see
the variable declarations, function definition, utility functions,
etc. -- all without moving the code\footnote{As an aside, you probably
  want to use something like {\tt follow-mode} in emacs, or comparable
  technologies, so that you can make the most of your screen to see
  you can see as much code on screen as possible.  Also, when reading
  code it is essential that it is all ``hyperlinked''.  You can use
  webpages like {\tt lxr.linux.no}, or {\tt etags} to quickly jump to
  function definitions.}.

That said, compact code can have its own issues.  You want to use
blank lines.  You simply want to use them well.  The analogy to text
is appealing: When do you create a separate paragraph?  There are few
hard and fast rules concerning this, and it often comes down to style.
A rule of thumb is often ``use paragraphs to separate thoughts''.  A
similar guide can be used with code: add line breaks when the new code
is in some way a separate thought than the previous.  For example,
setting up some data-structure, and doing some error checking on it,
is a separate thought from doing the same for a different
data-structure.  The goal here is to make your code easier to read.
Not more beautiful.  Use line breaks to tell the reader ``this is
something different''.

\head{On using brackets as line breaks.}  Some programmers use
brackets in a similar manner to line breaks.  That is, they will do
the following:

    \begin{minipage}{3in}
      \footnotesize
      \lstset{language=C}
      \begin{lstlisting}
void *malloc(int sz)
{
    if (!sz)
    {
        return NULL;
    }
    ...
}
      \end{lstlisting}
    \end{minipage}

instead of

    \begin{minipage}{3in}
      \footnotesize
      \lstset{language=C}
      \begin{lstlisting}
void *malloc(int sz) {
    if (!sz) {
        return NULL;
    }
    ...
}
      \end{lstlisting}
    \end{minipage}

Notice the brackets are essentially acting as line breaks.  The same
question as before applies here: are these brackets signifying ``a
different thought''?  I don't believe all brackets are the same here.
The bracket after the function prototype {\em is} signifying a change
of thoughts.  It emphasizes the function prototype independent of the
function's code.  This is a useful function.  In \composite , the
bracket after function prototype is on a separate line.  However, for
conditionals, loops, and all other brackets used inside of a function,
they do {\em not} signify a different thought.  In fact the code
inside of the condition is intimately interrelated with the condition
itself -- it is the same thought.  Thus, in \composite , we do not put
the bracket on a different line after conditional, loops, etc...

\head{\composite\ conventions.} There are some rules that I follow,
and that you should follow if modifying the \composite\ code-base.
These include:
\begin{enumerate}[itemsep=0pt,topsep=1pt,parsep=1pt,leftmargin=2.5em]
\item After the variable declarations at the top of a block, always
  insert a line-break.
\item Before the final return statement in a function, always insert a
  line-break.
\item After clusters of error checking, insert a line break.
\item Put open brackets on a separate line after function prototypes,
  but not on a separate line after conditionals and loops.
\end{enumerate}

\head{Indentation.}  First, I'll start off by repeating, there is no
correct indentation style.  Space, tabs, 2 spaces, 4 spaces, 8 spaces
-- there is no correct answer.  However, I have a preference, and
reasons for that preference that might help inform your own
preferences.  Goal 2 is to have formatting aid the reading process.  I
believe that the question of indentation is so subjective, that all
arguments I've heard for it aiding reading are unconvincing.  If
you're used to a specific indentation, then it will be easier to read
code using that indentation.  Thus, I focus on Goal 1: can formatting
itself encourage good practices?

I will make an assumption, largely motivated by my own experience:
code with many levels of indentation becomes hard to read.  This is
evident in Section~\ref{s:exception}.  This is largely because it is
more difficult to follow the main path of the code.  Often this
indentation is unavoidable (especially when writing algorithms) as
loops are required.  However, as you get deeper than 4 levels of
indentation, code simply becomes more difficult to parse: you must
remember the context for where you are -- I'm in 2 loops and 2
conditionals, you must remember what the purpose of those conditionals
and loops are, and you are left with only (7 $\pm$ 2) - 4 other items
in your short term memory.  Indentation levels are simply a proxy for
``some state that you must remember you're to understand this code''.
My experience has confirmed this.  For example, when writing a
complicated algorithm for doing hierarchical min-cut on a graph (see
mpd.c in \composite ), the first time I wrote it, I grappled with the
complexity.  Realizing that I was in too deep, I simply moved some of
the inner loops and conditionals into appropriately named functions.
This enabled me (as the programmer, not just the reader!) to visually
abstract away chunks of the code, and focus on the problem areas more
easily.

If we want to discourage high indentation levels, how can we do that?
Simple: make indentation more ``heavy weight''.  Thus in \composite ,
I define each level of indentation to be 8 spaces.  This is consistent
with Linux.  When you get to 4 levels of indentation, you visually
want to avoid any more by moving logic to utility functions.  It is a
visual cue that your code might be getting out of hand.  Of course,
there are exceptions.  The code in these exceptions is often a little
ugly as it stretches across the screen.  These conventions optimize
for the 99.9\% of cases out there that will aid in writing and reading
code.  It is reasonable to accept that 0.1\% might be formatted
sub-optimally.

\head{Consistency.}  Perhaps the most important stylistic advice w.r.t
formatting is to simply be consistent.  Whichever formatting style you
use, use it pervasively throughout a code-base.  If you work on a
large team, and this becomes difficult, then use {\tt indent} as part
of the build/check-in process.  Readers rely on conventions to
understand which parts of the code are important, and which aren't.
If you're schizophrenic about your formatting, you are denying readers
this important tool toward understanding your code.

\section{{\tt assert}s, {\tt BUG}s, and Error Conditions}
\label{s:exceptions}

Functions go wrong.  The arguments to the function might not be valid,
the assumed structure of a data-structure (see {\em invariants} in
Section~\ref{s:comments}) might be inconsistent, or many other
possible exceptional cases might occur.  Different types of exceptions
should be dealt with using different mechanisms.

\head{{\tt assert}ion usage.}  Often the inputs to a functions are
assumed to have some properties.  For example, perhaps pointers should
not be {\tt NULL}, different arguments must be initialized in a
certain manner, or should be related to each other in specific ways,
or numerical arguments should be within allowed ranges.  If such a
condition is not true, the question is this: is this an error within
your logic?  If it is simply an error, then you should use {\tt
  assert} statements to verify the truthfulness of the constraints.

You can get rid of assertions using preprocessor macros (by undefining
debugging), so you should not put functions that have side-effects
that must be executed for program correctness inside assertions
(because they might be compiled away!).  However, we have not observed
more than a 2\% performance degradation for including assertions, so
we suggest using them even in production code.

In systems, it is very important to differentiate between constraints
that must be true concerning input values that are passed from the
same ``source'' as the currently executing function.  For example, if
you are implementing a data-structure manipulation function in the
kernel, if the arguments come from other functions within the kernel,
it is appropriate to use assertions on the arguments.  In such a case,
if the invariant is violated, you'll find out immediately
(fail-stop).  {\em However}, if the arguments come from code with
different trust properties (i.e. user-level), you do {\em not} want to
assume the inputs are valid.  Do not use {\tt assert}ions here.

\head{Exceptional conditions, error codes, and return values.} As
pointed out in the previous bullet, if the input values do not come
from a trusted source, then you should not {\tt assert} on them (such
assertions would provide a trivial vector for users to crash a
component, or the kernel), and instead appropriate return values
should be synthesized.  See Section~\ref{s:exception} for mechanisms
to structure your code to handle such erroneous conditions.  If your
function is returning an integer (the common case), and negative
values are not return values within the defined range, then you should
return a {\tt -EVAL}, where {\tt EVAL} is defined in {\tt errno.h}.
For example, if the error is the result of an invalid argument, you
can return {\tt -EINVAL}, and if you run out of memory, return {\tt
  -ENOMEM}.  Checking for the occurrence of an error in the calling
function is often done by checking for a negative return value.

Often the error condition is not actually an error, and is instead
part of the interface.  For example, returning {\tt -EAGAIN} specifies
that ``data is not ready now, please try again later'' is used on
non-blocking interfaces.  This is an {\em exceptional} condition, but
not an {\em erroneous} condition.

\head{{\tt BUG} usage.} {\tt BUG()} will kill the current thread by
accessing the {\tt NULL} pointer (just as {\tt assert} does).  This is
done unconditionally.  {\tt BUG} statements will not be compiled away
even if debugging is not enabled.  This is not for checking conditions
on inputs, and is instead of stopping the system when something is
verifiably wrong.  This is similar to {\tt panic()} in other systems
such as Linux.

\section{Comments}
\label{s:comments}

A few notes on notation, and comment style.  These are largely
convention, and there isn't much of a justification for this style
over alternatives.  In \composite , we simply have standardized around
these rules.
\begin{itemize}[itemsep=0pt,topsep=1pt,parsep=1pt,leftmargin=2.5em]
\item {\tt /* TODO: ... */} This comment denotes a message about a
  functionality that should be added.  This functionality is not
  necessary for correct system operation, but might result in
  sub-optimal performance or inconvenient functionality.  This is work
  to be done in the future, and work that is not essential.
\item {\tt /* FIXME: ... */} This comment denotes a message about
  functionality, or logic that must be augmented for correct
  functionality.  If, for example, an edge case is not implemented, a
  {\tt FIXME} should be added to document the lack of this edge case
  logic.  In an ideal world, we'd write code that deals with every
  case.  However, this is often not possible due to the realities of
  software development.  Please minimize the number of {\tt FIXME}s,
  and use only where appropriate.
\item {/* ... */} {\em vs.} {//}.  We simply prefer the use of the
  former over the latter.  For multi-line comments, we use the
  formatting:

    \begin{minipage}{3in}
      \footnotesize
      \lstset{language=C}
      \begin{lstlisting}
/*
 * Multi-line
 * comment
 */
      \end{lstlisting}
    \end{minipage}

  Ascii-art in multi-line comments to diagram data-structures is
  welcomed, and the emacs {\tt artist-mode} can be quite useful in
  producing such diagrams.

\item {\em Pre-conditions, post-conditions, and invariants}.  Often
  during the design process, and later in development, it is useful to
  employ contract-driven development.  Often this is encoded in
  assertions, which is the optimal solution, but occasionally, it is
  not realistically possible to include the assertion due to the cost
  of the checks (i.e. on fast paths).  In these cases, you should
  still document the conditions that must exist before the call of the
  function (pre-conditions), the conditions that must be true after
  the function executes (post-conditions), and the invariants often
  imposed on data-structures (e.g. trees must be acyclic).  In such a
  case, it is useful to include a comment before the function.  We use
  the function to add a child to a tree as an example:

    \begin{minipage}{3in}
      \footnotesize
      \lstset{language=C}
      \begin{lstlisting}
/*
 * preconditions: p != NULL, c != NULL, c.parent == NULL
 * postconditions: is_child(p, c), c.parent == p
 * invariants: tree is acyclic.
 */
int tree_add_child(struct tree_node *p, struct tree_node *c);
      \end{lstlisting}
    \end{minipage}
\end{itemize}

  It is also acceptable to abbreviate ``preconditions'' as ``pre'',
  and ``postconditions'' as ``post''.

\section{Function Visibility, Location, and Modifiers}
\label{s:visibility}

A practical question you must answer when using C, is what should a
function's visibility be?  A function's visibility is determined by a
number of factors:
\begin{enumerate}[itemsep=0pt,topsep=1pt,parsep=1pt,leftmargin=2.5em]
\item Is the function {\tt static}?
\item Is the function implemented in a header file ({\tt *.h}), or a
  {\tt *.c} file?
\end{enumerate}

How do these decisions change software development characteristics,
and efficiency?  First, I'm going to assume that we're using {\tt gcc}
as your compiler, and not a compiler that does link-time inlining
(such as {\tt llvm}).  When you implement a function in a {\tt *.c}
file, it will generate an object file that will be linked together
with other objects.  When linking together objects, calls to that
function will have the proper function address inserted into the
executable.  When a function is implemented in a {\tt *.h} file, that
function becomes part of the same object as the functions that call
it.  There are multiple implications:
\begin{itemize}[itemsep=0pt,topsep=1pt,parsep=1pt,leftmargin=2.5em]
\item {\tt gcc} does not inline functions across objects.  You will
  always have a function all overhead when the function is compiled
  into a separate object.
\item When a function is in the same object (i.e. compiled in the same
  compilation unit), then it can be inlined according to the
  compiler's heuristic.
\end{itemize}

When a function is marked as {\tt static}, the function is not visible
outside of this object (compilation unit).  You can check this by
looking at the output of {\tt nm object.o}; functions with lower case
``symbol type'' are not visible outside of the object.  Thus, a
function inside another object cannot call that function.  If the
function is not {\tt static}, then {\tt nm} will yield a capital-case
symbol type, which denotes that it is visible from other objects:
functions within another compilation unit can call that function.

A note: a function defined in multiple object files that are linked
together cannot be visible outside of either object, or else the
linker will fail with an error.  This is the equivalent of defining
the function twice!  No compiler can deal with this case well.

The question is how to decide for any function, where to implement it,
and what modifiers to use?  Lets go through the conditions:
\begin{itemize}[itemsep=0pt,topsep=1pt,parsep=1pt,leftmargin=2.5em]
\item {\tt static} and {\tt *.h}: This is a common usage for
  performance-sensitive code.  The function being {\tt static} means
  that it can be defined in this compilation unit, and it won't
  conflict if it is also used in another compilation unit and the
  objects are linked together.  The {\tt inline} modifier is often
  used as well to suggest that the compiler inline the function.  This
  is generally the only reason to implement functions in this way.

  You must really know what you're doing with this implementation
  style.  If you implement functions in this style indiscriminately,
  you can end up using quite a bit of memory because when functions
  are inlined, their contents are duplicated at all call-points.
\item {\tt static} and {\tt *.c}: By default, this should be the way
  that you implement functions.  Most functions are not called so
  frequently from outside of their object that they should be in
  header files, and you want to avoid any naming conflicts with
  functions with the same name in other objects, so static makes
  sense.
\item Not {\tt static} and {\tt *.h}: Don't do this.  There are uses,
  but they are sufficiently rare, that I'll simply say: don't do this.
  You are asking for link-time errors where the same header file is
  included in two {\tt *.c} files, and they are linked together.
\item Not {\tt static} and {\tt *.c}: If you have a function that you
  want to be called from another {\tt *.c} file, this is the default.
  You will often also have a {\tt *.h} file that includes {\tt only}
  the prototype of the function.  That will enable the other {\tt *.c}
  file to compile, as it sees the prototype, and upon linking, it can
  now call the function.

  You can safely think about these functions as defining the external
  {\em interface} into the object.
\end{itemize}

\head{Example.}  A good example of the split of functionality into
     {\tt .c} and {\tt .h} files is the {\tt cbuf} implementation.
     The functions in {\tt src/components/include/cbuf.h} are
     performance critical, so they should be inlined into the calling
     code.  In contrast, the functions in {\tt
       src/components/interface/cbuf\_c/cbuf\_c.c} are also compiled
     into the same components and define the rest of the {\tt cbuf}
     interface.  Specifically, you can see that the functions in the
     {\tt .h} files attempt to be small, and call out to ``slow
     paths'' that are implemented in the {\tt .c} file.  This avoids
     memory waste by inlining all of the functionality, and makes the
     ``fast path'' as fast as possible.

%\subsection{Formatting in \composite}

%% \section{Object Orientated Programming in C}
%% \label{s:oo}


%% \ref{s:naming}.

\newpage
\section{Appendix: Style Rules for Systems@GWU}
\label{s:appendix}

A summary of the coding rules in \composite\ follows.  I don't provide
justifications for these rules as the previous sections cover that.
Many of these are adapted from the BSD ({\tt man 9 style}) and Linux
Style guides ({\tt Documentation/CodingSyle} in Linux).  Note that
many of the rules line up with the BSD coding style well.

\subsection{Comments}

\begin{itemize}
\item Do not use comments of the style

    \begin{minipage}{3in}
      \footnotesize
      \lstset{language=C}
      \begin{lstlisting}
// comment
      \end{lstlisting}
    \end{minipage}
\item Single-line comments:

    \begin{minipage}{3in}
      \footnotesize
      \lstset{language=C}
      \begin{lstlisting}
/* single-line comment */
      \end{lstlisting}
    \end{minipage}
\item Multi-line comments:

    \begin{minipage}{3in}
      \footnotesize
      \lstset{language=C}
      \begin{lstlisting}
/*
 * Multi-line comments use
 * a structure like this
 */
      \end{lstlisting}
    \end{minipage}
\item Comments at the head of functions to explain their high-level
  functionality, and comments that should be emphasized:

    \begin{minipage}{3in}
      \footnotesize
      \lstset{language=C}
      \begin{lstlisting}
/**
 * Emphasized comments use a structure like this.
 * More ``*'' entail more emphasis.
 */
      \end{lstlisting}
    \end{minipage}
\item Comments do not have to be alone on a line, and can follow code,
  so long as this does not create too long a line.
\item Note that in {\tt emacs}, {\tt M-;} will create an empty comment
  for you of the single-line style.
\end{itemize}

\subsection{Indentation and Spacing}

\head{Indentation.} Use an indentation level of 8 spaces.  Your code
will get ugly if it is indented more than 3 levels.  This is
intentional and means that you should not do that.  If your code
indents more than 3 levels, move some of it to a (well-named)
sub-function, or handle your error handling correctly.

\head{Function spacing.} Function definitions should take the form:

    \begin{minipage}{3in}
      \footnotesize
      \lstset{language=C}
      \begin{lstlisting}
static int
sched_block(spdid_t spdid)
{
        /* contents */
}
      \end{lstlisting}
    \end{minipage}

Return values and qualifiers/attributes are on a separate line, and
function names are always at the same level of indentation so they are
easy for the eye to find.

The only exception to this for functions whose body is a single
statement:

\head{Function spacing.} Function definitions should take the form:

    \begin{minipage}{3in}
      \footnotesize
      \lstset{language=C}
      \begin{lstlisting}
static int
sched_block(spdid_t spdid)
{ /* single statement */ }
      \end{lstlisting}
    \end{minipage}

Function prototypes can take a less stringent form if they can fit
onto a single line:

    \begin{minipage}{3in}
      \footnotesize
      \lstset{language=C}
      \begin{lstlisting}
static int sched_block(spdid_t spdid);
      \end{lstlisting}
    \end{minipage}

Please specify the name of each of the arguments, not just their
types.

\head{Conditional/loop spacing.} Spaces/newlines should be used as
follows in conditionals that have contents of a single line, or
multiple lines:

    \begin{minipage}{3in}
      \footnotesize
      \lstset{language=C}
      \begin{lstlisting}
if (expr0 && expr1) {
        /* contents */
}
      \end{lstlisting}
    \end{minipage}

Comparably, {\tt for} loops should be formatted as:

    \begin{minipage}{3in}
      \footnotesize
      \lstset{language=C}
      \begin{lstlisting}
for (exprinit ; exprsentinel ; exprincr) {
        /* contents */
}
      \end{lstlisting}
    \end{minipage}

Use a space after these keywords: {\tt if, switch, case, for, do,
  while}, but not with {\tt sizeof, typeof, alignof,
  \_\_attribute\_\_}.  Note the space before {\em and} after the {\tt
  ;}.  This is to visually emphasize the transition between sections.

The only exception to this spacing, for both conditionals and loops is
when the contents can fit onto the same line as the conditional or
loop.  In that case, the following is acceptable.  This is the {\em
  only} case where the enclosing {\tt \{} and {\tt \}} are not
required.

    \begin{minipage}{3in}
      \footnotesize
      \lstset{language=C}
      \begin{lstlisting}
if (expr0 && expr1) stmt;
      \end{lstlisting}
    \end{minipage}

This is commonly used in the following scenario:

    \begin{minipage}{3in}
      \footnotesize
      \lstset{language=C}
      \begin{lstlisting}
if (expr0) stmt0;
else       stmt1;
      \end{lstlisting}
    \end{minipage}

If an {\tt if} branch or an {\tt else} branch is multiple lines, then
they both must have brackets.  It is OK to use the following style
with {\tt \} else \{} on the same line:

    \begin{minipage}{3in}
      \footnotesize
      \lstset{language=C}
      \begin{lstlisting}
if (expr0) {
        /* contents */
} else {
        /* contents */
}
      \end{lstlisting}
    \end{minipage}


\head{Breaking up lines.}  If the expressions constituting the
proposition in a conditional do not fit well on a single line, then
break them up appropriately onto separate lines, and make sure that
the second line is indented to match its enclosing context.  Two
examples:

    \begin{minipage}{3in}
      \footnotesize
      \lstset{language=C}
      \begin{lstlisting}
if (some_complicated_function(a) && some_other_thing(b) &&
    a_really_long_function_name(c)) {
      /* contents */
}
      \end{lstlisting}
    \end{minipage}

or

    \begin{minipage}{3in}
      \footnotesize
      \lstset{language=C}
      \begin{lstlisting}
if (a_really_long_function_name(some_function(a), some_other_fn(b),
                                third_function(c))) {
        /* contents */
}
      \end{lstlisting}
    \end{minipage}

\head{Aligned expressions.} As shown above, the two statements are
vertically lined up.  Try and do this uniformly throughout your code.
It is especially relevant for assignments.

\head{Example.}  Many of the above rules can be seen below.

    \begin{minipage}{4in}
      \footnotesize
      \lstset{language=C}
      \begin{lstlisting}
struct cbufp_bin *bin;

cbi = malloc(sizeof(struct cbufp_info));
if (!cbi) goto done;

/* Allocate and map in the cbuf. */
cbid             = cmap_add(&cbufs, cbi);
cbi->cbid        = cbid;
size             = round_up_to_page(size);
cbi->size        = size;
cbi->owner.m     = NULL;
cbi->owner.spdid = spdid;
INIT_LIST(&cbi->owner, next, prev);
INIT_LIST(cbi, next, prev);

bin = cbufp_comp_info_bin_get(cci, size);
if (!bin) bin = cbufp_comp_info_bin_add(cci, size);
if (!bin) goto free;

if (bin->c) ADD_LIST(bin->c, cbi, next, prev);
else        bin->c = cbi;
if (cbufp_alloc_map(spdid, &(cbi->owner.addr),
		    (void**)&(cbi->mem), size)) goto free;
      \end{lstlisting}
    \end{minipage}

\head{Empty lines.}  It is necessary to use blank lines to break up
different ``thoughts'' in your code, just as we use paragraphs in
text.  As your function transitions between different phases (error
checking, setting up the environment, changing state in structure A,
then structure B, and preparing to return), use blank lines to
visually denote the change.  The following blank lines are mandatory:
\begin{itemize}
\item After your variable declarations in a block.
\item After clusters of error checking, insert a line break.
\item Before the return statement at the end of the function.
\end{itemize}

\head{{\tt goto} labels.} {\tt goto} labels should be on a line on
their own.  They should often be named either {\tt done}, {\tt err},
or {\tt retry}, depending on their function.  {\tt retry} labels
should be used very infrequently.

\subsection{Error Conditions}

You {\bf must} have pervasive error checking.  If you call a function,
you should understand how it passes errors to the caller.  You should
have a conditional after {\em every} function call that can have an
error to check for that error.  If you do not do this, not only will
your code never be merged, but you'll waste days of your own debugging
time in the future.

\head{Error values.} If a function cannot have any error (i.e. it does
very little interesting including any memory allocation), then it can
return {\tt void}.  If a function returns an {\tt int}, then any
negative value denotes an error.  Thus conditionals checking {\tt if
  (fn(a) < 0) ...} are common.  To provide a more descriptive error
code, you can use a value from {\tt errno.h}, for example returning
{\tt -ENOMEM} if memory allocation failed, or {\tt -EINVAL} if a
passed in argument was invalid.  It is not uncommon that a calling
function won't change its logic depending on the {\em type} of
failure, but will need to pass that failure condition down to its
caller.  The pattern for dealing with this is:

    \begin{minipage}{3in}
      \footnotesize
      \lstset{language=C}
      \begin{lstlisting}
if ((ret = fn(a))) return ret;
      \end{lstlisting}
    \end{minipage}

If a function returns a pointer, then {\tt NULL} is the default return
value.  If failure can occur for multiple reasons, then the function
should return an {\tt int}, and the pointer should be returned in a
function pointer argument.

\head{Error handling.} If an error is encountered, you often have to
unwind state that was created previously in the function (taking
locks, allocating memory, etc...).  This should be dealt with using
local gotos (equivalent to function-local exceptions in C++/Java).
See the above code for an example.  The labels being jumped to are
name corresponding to the action they will perform.  Above, the label
{\tt free} will free the {\tt cbi} that has been allocated.  The
generic label {\tt done} is used when there is no clean-up to do.

Often the function must return an error value.  The pattern for this
is that a {\tt ret} variable is created in the function of the
appropriate return type.  The {\tt return} at the end of the function
simply returns {\tt ret}.  For the successful path, {\tt ret} is set
to the non-error value (e.g. {\tt 0}).  In each condition, it is set
to the appropriate value.  For example:

    \begin{minipage}{3in}
      \footnotesize
      \lstset{language=C}
      \begin{lstlisting}
if (fn(a) < 0) {
        ret = -EINVAL;
        goto done;
}
      \end{lstlisting}
    \end{minipage}

See examples of this in the last example in
Section~\ref{ss:exceptions}.

Given this is a common pattern, you can instead use the {\tt
  cos\_throw(ret\_val, label)} macro defined in {\tt cos\_component.h}.

    \begin{minipage}{3in}
      \footnotesize
      \lstset{language=C}
      \begin{lstlisting}
if (fn(a) < 0) cos_throw(-EINVAL, done);
      \end{lstlisting}
    \end{minipage}

\subsection{Macros}

For multi-statement macros, wrap them in an enclosing {\tt do \{ ... \}
  while(0)} and always align the trialing {\tt $\backslash$} as in:

    \begin{minipage}{3in}
      \footnotesize
      \lstset{language=C}
      \begin{lstlisting}
#define FOO(a, b) do {              \
        something((a));             \
        something_else((b));        \
} while (0)
      \end{lstlisting}
    \end{minipage}

Follow all common macro guidelines and don't write complicated macros
unless you know what you're doing.  Even then, try and avoid it.

Macros are always capital.

\subsection{Naming, Variables, Functions, and Types}

Functions should be named in a {\tt subject\_verb} style, especially
in inter-component interfaces.  {\tt subject} in this case is the
namespace of the family of functions.  You have more leeway in naming
{\tt static} functions as they are only visible in the current file.
All functions that are intended to be used only in this file
(e.g. helper functions) must be static.

\head{Variables.} Within a function, do not mix your declarations with
code.  Declare all the function's variables at the head of the
function.  If a variable is only used in an enclosing scope, then you
can define it there.  It is important for a reader to be able to find
type declarations easily, and this rule aids that.

Place the declaration of a variable in the most restricted set of {\tt
  \{...\}}.  If the variable can be declared at the start of a
conditional/loop block, do so.

The cases within {\tt switch} statements often want to contain local
variables.  In such cases, adding braces is appropriate.  For example:

    \begin{minipage}{3in}
      \footnotesize
      \lstset{language=C}
      \begin{lstlisting}
switch(var) {
case VAR_CASE1:
        return 0;
case VAR_CASE2:
{
        int used_locally;

        used_locally = blah();
        /* ... */

        return used_locally;
}
default:
        return -EINVAL;
}
      \end{lstlisting}
    \end{minipage}

When a pointer is declared, the {\tt *} should bind to the variable
name, not the type as in:

    \begin{minipage}{3in}
      \footnotesize
      \lstset{language=C}
      \begin{lstlisting}
int *foo, *bar;
      \end{lstlisting}
    \end{minipage}

A commonly used pattern is to abbreviate the type for the name of a
local variable.  See {\tt cbi} above in the verbose example which is a
variable of the type {\tt struct cbuf\_info}.  Though {\tt cbi} isn't
descriptive, it is easy to look at the header of the function, and see
the type, and maintain an association between the two via the name.

\head{Type naming and usage.} All {\tt struct}s and types should be
defined at the top of the file or in a header file (if they are shared
amongst files, i.e. visible across an API).  {\tt typedef}s should
{\em only} be used when the type they are creating can be passed in a
register, and they should be named with a trailing ``{\tt \_t}'' to
denote ``type''.  It is important to be able to look at the type of a
variable and understand its calling conventions.  Importantly, never
pass a {\tt struct} or {\tt union} by value.  Always pass it by
reference.  If the variable is a {\tt typedef}, or a primitive type,
then you can pass by value.

{\tt enum} types should have an associated {\tt typedef}, and all
values are upper-case.  For example:

    \begin{minipage}{3in}
      \footnotesize
      \lstset{language=C}
      \begin{lstlisting}
typedef enum {ONE, TWO} enumtype_t;
      \end{lstlisting}
    \end{minipage}

Naming in many cases is dependent on taste, so I'm not going to
include huge numbers of rules here.

\subsection{Constant instantiation.}  Favor hex-decimal
representations for values that are not primarily thought of as digit
values.  For power-of-two values, favor using shifted constants to
make it explicit which bit is set.  For example, here is a typical
flag (bit-field) definition:

    \begin{minipage}{3in}
      \footnotesize
      \lstset{language=C}
      \begin{lstlisting}
typedef enum {
        THD_PREEMPTED = 1,
        THD_RUNNING   = 1<<1,
        THD_WAITING   = 1<<2,
} thd_flags_t;
      \end{lstlisting}
    \end{minipage}

For a set of constant values from which only a single value can be
used at a time, rely on {\tt enum} naming, unless the specific values
are important.  Always use a {\tt \_MAX}, or {\tt \_MAXVAL} enum value
to designate the maximum possible value for boundary checking.

    \begin{minipage}{3in}
      \footnotesize
      \lstset{language=C}
      \begin{lstlisting}
typedef enum {
        THD_KERNEL,
        THD_USER,
        THD_MN,
        THD_MAXVAL
} thd_type_t;
      \end{lstlisting}
    \end{minipage}

...and later...

    \begin{minipage}{3in}
      \footnotesize
      \lstset{language=C}
      \begin{lstlisting}
void
foo(thd_type_t t)
{
        if (t >= THD_MAXVAL) return -EINVAL;
        ...
}
     \end{lstlisting}
    \end{minipage}


\subsection{Parenthesis}

If there is any ambiguity amongst the associativity of operators, just
use parenthesis.  If there is any question if you should use them,
then use them.

\subsection{Include Files}

All of your {\tt \#include}s should be at the top of the file.

{\tt .h} files should include all structures that are exposed by the
API of the compile unit, and the prototype of any functions in the
compile unit's API.  Files that must be {\tt inline}d can be defined
within the include file (as {\tt static inline}).  Very few functions
require this; keep this in mind.

\subsection{Visibility}

Visibility of global variables, and functions should be restricted to
the smallest possible unit.  Helper functions within a {\tt .c} file
should always be marked as {\tt static}, as with global variables.
Additionally, helper functions should often begin with {\tt \_\_} to
visually make explicit that this function is not part of the public
API (e.g. {\tt \_\_thread\_enqueue(...)}).

The ideal is that the function prototypes and {\tt static inline}
functions in the {\tt .h} file exactly match the symbols that are
visible in the {\tt .c}'s object (i.e. as seen with {\tt nm}).

Global variables should always be {\tt static} local variables if they
are only used within a single function.  This is rare.

The use of {\tt extern} should be minimized, but there are good
reasons to use it.  They often revolve around the desire in complicated
programs to split the logic into units, often in separate {\tt .c}
files.  It is sometimes cleaner for two {\tt .c} files to have direct
access to a global variable.  Given enough time and a willingness to
generalize the functionality, this usage can be hidden by a dedicated
{\tt .c/.h} file pair to manage the data-structure, but this is not
always the right thing to do.

\newpage
\section{Document History}
\label{s:hist}

\begin{itemize}
\item \lsthead{Version 0.1}, 7/29/11 -- Hacked out in a day.  Includes
\begin{itemize}[itemsep=0pt,topsep=1pt,parsep=1pt,leftmargin=2.5em]
\item on reading code,
\item most important style rule (\ie\ naming),
\item exceptional cases,
\item typedef usage,
\item macro usage,
\item indentation and formatting, and
\item plenty of errors, I'm sure.
\end{itemize}
\item \lsthead{Version 0.11}, 8/28/11 -- Small changes and added the
  onion model.
\item \lsthead{Version 0.12}, 7/11/12 -- Added sections on Comments
  (Section~\ref{s:comments}) and Exceptions/Errors
  (Section~\ref{s:exceptions}).
\item \lsthead{version 0.13}, 7/12/12 -- added section on function
  visibility, location, and modifiers, Section~\ref{s:visibility}.
\item \lsthead{version 0.2}, 3/6/13 -- added appendix section that
  summarizes many of the coding style rules
  (Section~\ref{s:appendix}).
\item \lsthead{version 0.21}, 9/10/15 -- added to/edited the appendix
  section that summarizes many of the coding style rules
  (Section~\ref{s:appendix}).
\end{itemize}

\head{TODO}
\begin{itemize}[itemsep=0pt,topsep=1pt,parsep=1pt,leftmargin=2.5em]
\item Edit and proof-read
\item OO in C
\item Makefiles and the build procedure
\end{itemize}

\end{document}
